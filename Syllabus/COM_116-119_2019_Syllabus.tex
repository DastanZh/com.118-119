\documentclass[12pt,a4paper,oneside]{article}

\usepackage[margin=3cm]{geometry}

\usepackage{hyperref}
\hypersetup{
    pdftitle={COM 116, 117, 118, 119 Structural Programming, Programming I, II},%
    pdfauthor={Toksaitov Dmitrii Alexandrovich},%
    pdfsubject={Syllabus},%
    pdfkeywords={COM;}{116;}{117;}{118;}{119;}{syllabus;}{programming;}{I;}{II;}{Java},%
    colorlinks,%
    linkcolor=black,%
    citecolor=black,%
    filecolor=black,%
    urlcolor=black
}

\newcommand{\R}[1]{\uppercase\expandafter{\romannumeral #1\relax}}

\begin{document}

    \title{COM 116, 117, 118, 119 Structural Programming, Programming \R{1}, \R{2}}
    \author{
        American University of Central Asia\\
        Software Engineering Program
    }
    \date{}
    \maketitle

    \section{Course Information}

        \begin{description}
            \item[Course ID]\hfill\\
                COM 116, 2967\\
                COM 117, 2968\\
                COM 118, 4322\\
                COM 119, 4357
            \item[Course Repository]\hfill\\
                \url{https://github.com/auca/com.116-119}
            \item[Class Discussions]\hfill\\
                \url{https://piazza.com/auca.kg/spring2019/com119}
            \item[Place]\hfill\\
                AUCA, room 410\\
                AUCA, laboratory G30, G31
            \item[Time]\hfill\\
                Lecture: Monday 12:45--14:00\\
                Lecture: Monday 14:10--15:25\\
                Lab: Monday 14:10--15:25\\
                Lab: Friday 12:45--14:00\\
                Lab: Friday 14:10--15:25\\
                Lab: Tuesday 10:50--12:05\\
                Lab: Tuesday 12:45--14:00\\
                Lab: Tuesday 14:10--15:25
        \end{description}

    \section{Contact Information}

        \begin{description}
            \item[Instructor]\hfill\\
                Shostak Dmitrii Grigorievich\\
                \href{mailto:shostak_d@auca.kg}{shostak\_d@auca.kg}\\
                Toksaitov Dmitrii Alexandrovich\\
                \href{mailto:toksaitov_d@auca.kg}{toksaitov\_d@auca.kg}
            \item[Teacher Assistants]\hfill\\
                Samuel Ramaley Furr\\
                \href{mailto:furr_s@auca.kg}{furr\_s@auca.kg}
            \item[Office]\hfill\\
                AUCA, room 315
            \item[Office Hours]\hfill\\
                Sat, Sun (remotely through Skype)\\
                Additional office hours are scheduled by the TA
        \end{description}

    \section{Course Overview}

        This course helps to equip students with basic skills needed for
        structural and object-oriented programming. At the completion of the
        course students should understand fundamental programming concepts such
        as flow control, objects, classes, methods, procedural decomposition,
        inheritance and polymorphism; be able to write simple applications using
        most of the capabilities of the Java programming language and apply
        principles of good programming practices throughout the process.  This
        course is designed for Software Engineering majors and minors.

    \section{Topics Covered}

        \begin{itemize}
            \item Introduction to the Process of Software Development
            \item Selections
            \item Loops
            \item Methods
            \item Single- and Multidimensional Arrays
            \item Objects and Classes
            \item Inheritance and Polymorphism
            \item Abstract Classes and Interfaces
            \item Exception Handling
            \item GUI and Computer Graphics Basics
            \item Generics and Container Classes
            \item Working with I/O
        \end{itemize}

    \section{Exams}

        \subsection{Lectures}

            Students will have to take midterm and final examinations on topics
            discussed during lectures. Each examination is in the form of a quiz
            with a set of open and multiple choice questions.

        \subsection{Labs}

            Students will have 8 laboratory tasks, get a number of problems from
            an Online Judge System, and have to finish two projects developing
            real-world applications. Students will have to defend their work to
            the instructor during separate midterm and final examination
            sessions.

    \section{Course Materials, Recordings and Screencasts}

        Students will find all the course materials on GitHub. We hope by
        working with GitHub students will become familiar with the Git version
        control system and the popular (among developers) GitHub service. Though
        version control is not the focus of the course, some course tasks may
        have to be submitted through it on the GitHub Classroom service.

        Every class is screen casted online and recorded to YouTube for
        students’ convenience. An ability to watch a class remotely MUST NOT
        be a reason to not attend the class. Active class participation is
        necessary to succeed on this course.

    \section{Reading}

        Introduction to Java Programming, Comprehensive, 10th Edition by Y.
        Daniel Liang (AUCA Library Call Number: QA76.73.J38 L5218 2011, ISBN:
        978-0132130806)

    \section{Grading}

        \subsection{Lectures}

            \begin{itemize}
                \item Midterm (14\%)
                \item Final (14\%)
            \end{itemize}

        \subsection{Labs}

            \begin{itemize}
                \item Labs 1--4 (16\%)
                \item Online Judge Problems (10\%)
                \item Project \#1 (10\%)\\\\
                    \textbf{Midterm Defense} (Labs + Online Judge Problems + Project \#1)\\
                \item Labs 5--8 (16\%)
                \item Online Judge Problems (10\%)
                \item Project \#2 (10\%)\\\\
                    \textbf{Final Defense} (Labs + Online Judge Problems + Project \#2)\\
            \end{itemize}

        \begin{itemize} \itemsep-10pt \parskip0pt \parsep0pt
            \item[--] 92\%--100\%: A\\
            \item[--] 85\%--91\%: A-\\
            \item[--] 80\%--84\%: B+\\
            \item[--] 75\%--79\%: B\\
            \item[--] 70\%--74\%: B-\\
            \item[--] 65\%--69\%: C+\\
            \item[--] 60\%--64\%: C\\
            \item[--] 55\%--59\%: C-\\
            \item[--] 50\%--54\%: D+\\
            \item[--] 45\%--49\%: D\\
            \item[--] 40\%--44\%: D-\\
            \item[--] Less than 40\%: F
        \end{itemize}

    \section{Rules}

        Students are required to follow the rules of conduct of the Software
        Engineering Department and American University of Central Asia.

        Team work is NOT encouraged. The same blocks of code or similar
        structural pieces in separate works will be considered as academic
        dishonesty and all parties will get zero for the task.

        Attendance is mandatory. More than three misses without a reason will
        result in 5 points being deducted from the student. If a student has
        health/family/personal emergency, he must notify the instructor if
        possible (through e-mail), to increases the chances for the miss to be
        not counted.

        Active work during the class may be awarded with up to 10 points at the
        instructor's discretion.

        Poor student performance during a class can lead to up to 3 points
        deducted from his final grade.

        Late submissions will receive a penalty of 10 points for every day after
        the deadline.

\end{document}
