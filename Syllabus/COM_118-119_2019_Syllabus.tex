\documentclass[12pt,a4paper,oneside]{article}

\usepackage[margin=3cm]{geometry}

\usepackage{hyperref}
\hypersetup{
    pdftitle={COM 118, 119 Structured Programming, Object-oriented Programming},%
    pdfauthor={Toksaitov Dmitrii Alexandrovich},%
    pdfsubject={Syllabus},%
    pdfkeywords={AUCA;}{COM;}{118;}{119;}{syllabus;}{structured;}{object-oriented;}{programming;}{Java},%
    colorlinks,%
    linkcolor=black,%
    citecolor=black,%
    filecolor=black,%
    urlcolor=black
}

\begin{document}

    \title{COM 118, 119: Structured Programming, Object-oriented Programming}
    \author{
        American University of Central Asia\\
        Software Engineering Department
    }
    \date{}
    \maketitle

    \section{Course Information}

        \begin{description}
            \item[Course Codes]\hfill\\
                COM 118\\
                COM 119
            \item[Course IDs]\hfill\\
                4322\\
                4357
            \item[Prerequisite]\hfill\\
                COM-119, Object-oriented Programming
                or
                COM-223, Algorithms and Data Structures
                COM-410, Computer Architecture and Organization
            \item[Credits]\hfill\\
                6
            \item[Professors, TAs, Time, Place]\hfill\\
                Lecture (Michael Brady): Monday 14:10--15:25, TBD\\
                Lecture (Michael Brady): Wednesday 14:10--15:25, TBD\\
                Lab (Dmitrii Shostak): Thursday 10:50--12:05, G31\\
                Lab (Dmitrii Shostak): Thursday 12:45--14:00, G31\\
                Lab (Dmitrii Shostak): Thursday 14:10--15:25, G31\\
                Lab (Dmitrii Shostak): Friday 12:45--14:00, G31\\
                Lab (Dmitrii Toksaitov): Wednesday 9:25--10:40, G30\\
                Lab (Dmitrii Toksaitov): Wednesday 12:45--14:00, G30\\
                Lab (Dmitrii Toksaitov): Friday 12:45--14:00, G30\\
                TA Consultations (Bektur Umarbaev): By appointment
            \item[Course Repository]\hfill\\
                \url{https://github.com/auca/com.118-119}
			\item[Class Discussions]\hfill\\
                \url{https://piazza.com/auca.kg/fall2019/com118}\\
                \url{https://piazza.com/auca.kg/spring2020/com119}
        \end{description}

    \section{Contact Information}

        \begin{description}
            \item[Professors]\hfill\\
                Michael Brady\\
                \href{mailto:brady_m@auca.kg}{brady\_m@auca.kg}\\
                Dmitrii Shostak\\
                \href{mailto:shostak_d@auca.kg}{shostak\_d@auca.kg}\\
                Dmitrii Toksaitov\\
                \href{mailto:toksaitov_d@auca.kg}{toksaitov\_d@auca.kg}
            \item[Teacher Assistant]\hfill\\
                Bektur Umarbaev\\
                \href{mailto:umarbaev_b@auca.kg}{umarbaev\_b@auca.kg}
            \item[Office]\hfill\\
                AUCA, room 315
            \item[Office Hours]\hfill\\
                By appointment throughout the work week\\
                Remotely through Skype on Saturday and Sunday from 12:00 to 18:00
        \end{description}

    \section{Course Overview}

        This course helps to equip students with essential skills needed for structured and object-oriented programming. At the completion of the course, students should understand fundamental programming concepts such as flow control, objects, classes, methods, procedural decomposition, inheritance, and polymorphism; be able to write simple applications using most of the capabilities of the Java programming language and apply principles of good programming practices throughout the process.

        At the end of the course student should be able to research, analyze, design, develop, and maintain functioning software systems in accord to the goals of the AUCA Software Engineering Department and the 510300 IT competency standard (OK 1–7, ИК 1–7, ПК 1–15).

    \section{Topics Covered}
    
        Structured Programming
        
        \begin{itemize}
            \item Week 1--2: Introduction to the Process of Software Development (6 hours)
            \item Week 3--5: Selections (9 hours)
            \item Week 6--9: Loops (9 hours)
            \item Week 10--13: Methods (9 hours)
            \item Week 14--16: Single- and Multidimensional Arrays (9 hours)
        \end{itemize}
        
        Object-oriented Programming

        \begin{itemize}
            \item Week 1--3: Objects and Classes (9 hours)
            \item Week 4--6: Inheritance and Polymorphism (9 hours)
            \item Week 7--8: Abstract Classes and Interfaces (6 hours)
            \item Week 9--10: Exception Handling (6 hours)
            \item Week 11--12: GUI and Computer Graphics Basics (6 hours)
            \item Week 13--14: Generics and Container Classes (6 hours)
            \item Week 15--16: Working with I/O (6 hours)
        \end{itemize}

    \section{Exams}

        \subsection{Lectures}

            Students will have to take a midterm and final examinations on topics discussed during lectures. Each exam is in the form of a quiz with a set of open and multiple-choice questions.

        \subsection{Labs}

            Students will have eight laboratory tasks, get a number of problems from an Online Judge System, and have to finish two projects developing real-world applications. Students will have to defend their work to the instructor during separate midterm and final examination sessions.

    \section{Course Materials, Recordings and Screencasts}
        
        Students will find all the course materials on GitHub. We hope that by working with GitHub, students will become familiar with the Git version control system and the popular (among developers) GitHub service. Though version control is not the focus of the course, some course tasks may have to be submitted through it on the GitHub Classroom service.

    	Every class is screencasted online and recorded to YouTube for students’ convenience. An ability to watch a class remotely MUST NOT be a reason not to attend the class. Active class participation is necessary to succeed in this course.

    \section{Reading}

        Intro to Java Programming, Comprehensive Version, 10th Edition by Y.
        Daniel Liang (AUCA Library Call Number: QA76.73.J38 L5218 2011, ISBN-13:
        978-0133761313, ISBN-10: 0133761312)

    \section{Grading}

        \subsection{Lectures}

            \begin{itemize}
                \item Attendance/participation (5\%)
                \item Midterm (10\%)
                \item Final (15\%)
            \end{itemize}

        \subsection{Labs}

            \begin{itemize}
                \item Labs 1--4 (16\%)
                \item Online Judge Problems (10\%)
                \item Project \#1 (9\%)\\\\
                    \textbf{Midterm Defense} (Labs + Online Judge Problems + Project \#1)\\
                \item Labs 5--8 (16\%)
                \item Online Judge Problems (10\%)
                \item Project \#2 (9\%)\\\\
                    \textbf{Final Defense} (Labs + Online Judge Problems + Project \#2)\\
            \end{itemize}
 
        \subsection{Total}

            \begin{itemize}
                \item 100\% is formed from lecture exams (30\%) and lab exams (70\%).
            \end{itemize}

        \subsection{Scale}
    
            \begin{itemize} \itemsep-10pt \parskip0pt \parsep0pt
                \item[--] 92\%--100\%: A\\
                \item[--] 85\%--91\%: A-\\
                \item[--] 80\%--84\%: B+\\
                \item[--] 75\%--79\%: B\\
                \item[--] 70\%--74\%: B-\\
                \item[--] 65\%--69\%: C+\\
                \item[--] 60\%--64\%: C\\
                \item[--] 55\%--59\%: C-\\
                \item[--] 50\%--54\%: D+\\
                \item[--] 45\%--49\%: D\\
                \item[--] 40\%--44\%: D-\\
                \item[--] Less than 40\%: F
            \end{itemize}

    \section{Rules}

        Students are required to follow the rules of conduct of the Software Engineering Department and the American University of Central Asia.

    \subsection{Participation}
    
        Active work during the class may be awarded with up to 5 extra points at the instructor’s discretion.

        Poor student performance during a class can lead to up to 5 points being deducted from the final grade.

        Instructors may conduct pop-checks during classes at random without prior notice. Students MUST be ready for every class in order not to lose points.

    \subsection{Attendance}
    
        Lecture classes has 5\% attendance/participation part.
        
        More than three misses on practice classes without reason will result in 10 points being deducted from the student for every day. If a student has health/family/personal emergency, he MUST notify the instructor in advance (e.g., through e-mail). The student MUST also provide valid proof afterward. Without prior notice and valid proof, the miss will still be counted.
        
    \subsection{Questions}
    
        We believe that a question from one student is most likely a question that other students are also interested in. That is why we encourage students to use Piazza to ask questions in public that other students can see and answer and NOT ask them through E-mail in private UNLESS the question itself is about private matters to discuss with the professor.

    \subsection{Late Policy}

        Late submissions and late exams are not allowed. Exceptions may be made at the discretion of the professor only in force-majeure circumstances.

    \subsection{Incomplete}
    
        As with late exams, the grade I may be awarded only in exceptional circumstances. The student must start a discussion on getting the grade I with the instructors in advance and not during the last week before the final exams.
        
    \subsection{Academic Honesty}
    
        Plagiarism can be defined as “an act or an example of copying or stealing someone else’s words or ideas and appropriating them as one’s own”. The concept of plagiarism applies to all tasks and their components, including program code, abstracts, reports, graphs, statistical tables, etc.

    	In addition to being unethical, this indicates that the student has not studied the given material. Tasks written from somewhere for 10\% or less will be assessed accordingly or will receive a 0 at the discretion of the teacher. If plagiarism is more than 10\%, the case will be transferred to the AUCA Disciplinary Committee.

        Students are not recommended to memorize before exams, as this is a difficult and inefficient way to learn; and since practice exams consist of open questions designed to test a student’s analytical skills, memorization invariably leads to the fact that the answers are inappropriate and of poor quality.
        
        On this course teamwork is NOT encouraged. The same blocks of code or similar structural pieces in separate submissions will be considered as academic dishonesty, and all parties will get zero for the task.

        The following are examples of some common acts of plagiarism:
        
        \begin{enumerate}
            \item Representing the work of others as their own
            \item Using other people's ideas or phrases without specifying the author
            \item Copying code snippets, sentences, phrases, paragraphs or ideas from
                  other people's works,published or unpublished, without referring to the author
            \item Replacing selected words from a passage and using them as your own
            \item Copying from any type of multimedia (graphics, audio, video, Internet 
                  streams), computer programs, graphs or diagrams from other people's works without representation of authorship
            \item Buying work from a website or from another source and presenting it as
                  your own work
        \end{enumerate}

\end{document}
