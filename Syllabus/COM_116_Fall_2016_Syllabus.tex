\documentclass[12pt,a4paper,oneside]{article}

\usepackage[margin=3cm]{geometry}

\usepackage{hyperref}
\hypersetup{
    pdftitle={COM 116, Programming I, Lectures},%
    pdfauthor={Toksaitov Dmitrii Alexandrovich},%
    pdfsubject={Syllabus},%
    pdfkeywords={COM;}{116;}{syllabus;}{programming;}{I;}{Java},%
    colorlinks,%
    linkcolor=black,%
    citecolor=black,%
    filecolor=black,%
    urlcolor=black
}

\newcommand{\R}[1]{\uppercase\expandafter{\romannumeral #1\relax}}

\begin{document}

    \title{COM 116, Programming I, Lectures}
    \author{
        American University of Central Asia\\
        Software Engineering Department
    }
    \date{}
    \maketitle

    \section{Course Information}

        \begin{description}
            \item[Course ID]\hfill\\
                COM 116, 2967
            \item[Course Repository]\hfill\\
                \url{https://github.com/auca/com.116}
            \item[Place]\hfill\\
                AUCA, room 410
            \item[Time]\hfill\\
                Tuesday 9:25\\
                Tuesday 12:45
        \end{description}

    \section{Contact Information}

        \begin{description}
            \item[Instructor]\hfill\\
                Toksaitov Dmitrii Alexandrovich\\
                \href{mailto:toksaitov_d@auca.kg}{toksaitov\_d@auca.kg}
            \item[Office]\hfill\\
                AUCA, room 315
            \item[Office Hours]\hfill\\
                Monday 12:45--14:45\\
                Tuesday 10:50--12:45, 14:00--16:00\\
                Wednesday 12:45--14:45\\
                Friday 14:00--16:00
        \end{description}

    \section{Course Overview}

        This course helps to equip students with basic skills needed for
        object-oriented programming. At the completion of the course students
        should understand fundamental object-oriented concepts such as object,
        class, method, inheritance and polymorphism; be able to write simple
        applications using most of the capabilities of Java and apply principles
        of good programming practices throughout the process. This course is
        designed for Software Engineering majors and minors.

    \section{Exams}

        Students will have to take midterm and final examinations. The
        examinations are based on topics discussed during lectures. The
        examinations will have the form of a quiz with a set of multiple choice
        and open questions.

    \section{Reading}

        Introduction to Java Programming, Comprehensive, 8th Edition by Y.
        Daniel Liang (AUCA Library Call Number: QA76.73.J38 L5218 2011, ISBN:
        978-0132130806)

    \section{Grading}

        \begin{itemize}
            \item Midterm (20\%)
            \item Final (20\%)
        \end{itemize}

        Grading information for laboratory classes is available at
        \url{https://goo.gl/3vJnTu}.

        \begin{itemize} \itemsep-10pt \parskip0pt \parsep0pt
            \item[--] 92\%--100\%: A\\
            \item[--] 85\%--91\%: A-\\
            \item[--] 80\%--84\%: B+\\
            \item[--] 75\%--79\%: B\\
            \item[--] 70\%--74\%: B-\\
            \item[--] 65\%--69\%: C+\\
            \item[--] 60\%--64\%: C\\
            \item[--] 55\%--59\%: C-\\
            \item[--] 50\%--54\%: D+\\
            \item[--] 45\%--49\%: D\\
            \item[--] 40\%--44\%: D-\\
            \item[--] Less than 40\%: F
        \end{itemize}

    \section{Rules}

        Students are required to follow the rules of conduct of the Software
        Engineering Department and American University of Central Asia.

        Team work is NOT encouraged. The same blocks of code or similar
        structural pieces in separate works will be considered as academic
        dishonesty and all parties will get zero for the task.

\end{document}
